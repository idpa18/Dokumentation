\documentclass[../main.tex]{subfiles}

Im zweiten Weltkrieg war die politische und sicherheitstechnische Lage der Schweiz angespannt. Umgeben von den Kriegsparteien Italien, Frankreich und Deutschland, musste die Schweizer Armeeführung eine Strategie ausarbeiten, um die Neutralität der Schweiz zu wahren. Die sogenannte Reduit-Strategie, welche den Rückzug der Soldaten in die Alpen beinhaltete, ist bekannt, es gibt viele Informationen darüber. Diese Arbeit soll aber auch die zahlreichen anderen Strategien der Schweizer Armee beleuchten.

Die Operationsbefehle des Schweizer Generals Henri Guisan aus dem zweiten Weltkrieg sind beim Eidgenössischen Departement für Verteidigung, Bevölkerungsschutz und Sport erhältlich. Im Rahmen dieser IDPA möchte unsere Gruppe, bestehend aus Dario Wigger, Joel Wälti, Seya Schmassmann und Stefanie Jäger, die Operationsbefehle analysieren, und daraus geografische Karten ableiten, welche auf einer Webseite informativ und interaktiv dargestellt werden sollen.

Die Webseite ist das Produkt, das aus dieser Arbeit resultiert. Das Ziel der Webseite ist es, den Besuchern über die Schweiz und ihre Strategie im zweiten Weltkrieg zu informieren. Die Webseite gliedert sich in verschiedene Bereiche. In einem Bereich werden die wichtigsten Entscheidungsträger der damaligen Zeit dargestellt. In einem weiteren Bereich erhalten die Besucher grafischen Einblick in eine Reduit-Festung der Schweiz. Der dritte Bereich umfasst eine Karte, auf welcher zwei Aspekte abgebildet werden. 
Der erste Aspekt ist die Visualisierung der Operationsbefehle. Auf der Schweizerkarte werden die Operationsbefehle mit ihren geografischen Eckpunkten dargestellt. Die Benutzer können sich durch die Operationsbefehle klicken, und können damit die verschiedenen Strategien vergleichen. 
Der zweite Aspekt ist die Darstellung der Verteilung der Festungen und Bunker, welche im zweiten Weltkrieg von der Schweizer Armee benutzt wurden.
