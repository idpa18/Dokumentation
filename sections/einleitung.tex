\documentclass[../main.tex]{subfiles}

Das Ziel dieser Arbeit ist, die Verteidigungsstrategie der Schweiz im zweiten Weltkrieg aufzuzeigen. Dabei wird ein Fokus auf die Reduit-Strategie gelegt. Über die Reduit-Strategie gibt es bereits viele Informationen, diese Arbeit soll aber auch die zahlreichen anderen Strategien der Schweizer Armee beleuchten.

Im Rahmen dieser IDPA hat unsere Gruppe, bestehend aus Dario Wigger, 
Joel Wälti, Seya Schmassmann und Stefanie Jäger, verschiedenste Informationen
aus den Operationsbefehlen, welche von General Henri Guisan während dem zweiten Weltkrieg
ausgestellt wurden, zusammengetragen und in einer Webseite informativ und interaktiv 
dargestellt.

Das Ziel der Webseite ist es, den Besuchern über die Schweiz im zweiten Weltkrieg zu informieren. Den Besuchern soll Einblick in einen Bunker erhalten, auf der Schweizerkarte die verschiedenen Verteidigungsstrategien betrachten, und die Verteilung der Festungen und Bunker ansehen können. Um den Besuchern der Webseite einen möglichst guten Einblick in einen Bunker zu gewähren, wurde die Festung Vitznau besucht, und Fotos gemacht.