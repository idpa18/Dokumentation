\documentclass[../main.tex]{subfiles}
Unser Projekt haben wir nach der Projektmethode IPERKA gegliedert. 
\subsection{Informieren} \label{Informieren}
\subsubsection{Operationsbefehle} \label{Operationsbefehle}
Bei der Recherche für unsere Arbeit stiessen wir immer wieder auf Zitate, welche von einem Dokument Names "Operationsbefehl Nr. 13" stammten.
Dies brachte uns darauf, nach dem vollen Inhalt dieses Dokumentes und den 12 davor zu suchen. 
Da wir auf unserer Suche im Internet nicht fündig wurden, beschlossen wir auf der Internetseite admin.ch nach einer Kontaktstelle zu suchen.
Wir wurden mit einem Formular fündig. Die füllten wir mit unserer Bitte und einer Erklärung der Beweggründe aus und erhielten innerst kurzer Zeit Antwort vom Eidgenössischen Departement für Verteidigung,
Bevölkerungsschutz und Sport VBS. Das VBS stelte uns die Operationsbefehle, 22 an der Zahl, aus dem zweiten Weltkrieg zur Verfügung.

Da wir diese Befehle in Form von eingescannten PDFs erhielten, teilten wir das Lesen und Zusammenfassen unter den Gruppenmitgliedern auf. 
Aus jedem Befehl wurden die Gruppen, deren Zusammensetzung und Position, Verschiebungen und generelle Informationen ausgelesen und in einem Word-Dokument zusammengefasst und auf einer Karte eingezeichnet.
So hatten wir alle wichtigen Informationen kurz und prägnant in einer digitalen, weiter verarbeitbaren Form.

\subsubsection{Wichtige Personen} \label{Wichtige Personen}
Um dem Besucher unserer Webseite noch etwas mehr Hintergrundinformationen zu liefern, haben wir uns entschieden, die wichtigsten Personen, die für die Strategie im zweiten Weltkrieg verwantwortlich sind, auf der Webseite darzustellen. Dazu haben wir uns im Internet über die Schweiz im zweiten Weltkrieg mit ihren Entscheidungsträgern informiert.

\subsubsection{Festung Vitznau} \label{Festung Vitznau}
Für uns war es wichtig, eine Festung aus dem zweiten Weltkrieg von innen zu sehen, damit wir uns ein Bild vom Leben der Soldaten während dem Krieg machen können.
Im Internet informierten wir uns über Festungen, die wir besuchen könnten. Wir stiessen auf die Festung Vitznau, die uns grosszügigerweise kostenlos einen Führer, Erich Schneider, zur Verfügung stellte. Während rund zwei Stunden durfen wir die Festung Vitznau besichtigen, Fotos machen und Fragen stellen.
Als Vorbereitung auf den Besuch hatten wir uns im Internet über die Festung informiert, und Fragen vorbereitet.

\subsubsection{Festungen} \label{Festungen}
Das Herzstück der Schweizer Verteidigungsstrategie im 2. Weltkrieg sind die zahlreichen Festungseinrichtungen verteilt in der ganzen Schweiz.
Schnell stellte sich uns die Frage, wo sich diese, früher noch geheimen Bunker, denn eigentlich befinden.
Eine Suche nach einer Karte im Internet verlief erfolglos. Folglich kam dann die Idee, selbst eine Karte zu erstellen.
Mit Hilfe einer Liste der Bunker von Wikipedia, der Webseite "festung-oberland.ch" und Google Maps konnten schlussendlich eine Karte ausarbeiten.

\subsection{Planen} \label{Planen}
Zu Beginn des Projektes mussten wir einen Zeitplan erstellen, was wir mit Excell gemacht haben. Die Verfeinerung mit der Aufteilung der Aufgaben haben wir mit dem Onlinetool Clickup.com vorgenommen.

\subsection{Entscheiden} \label{Entscheiden}
Nach Abschluss der Information-Phase mussten wir Entscheide über den Inhalt unserer praktischen Arbeit treffen. Wir entschieden uns dazu, die Webseite in drei Teile aufzuteilen: Die wichtigsten Personen, eine ausführliche Ansicht der Festung Vitznau und die Schweizerkarte mit den Operationsbefehlen und Bunkern.

\subsection{Realisieren} \label{Realisieren}

\subsubsection{Dokumentation} \label{Dokumentation}
Die Dokumentation, die Sie gerade lesen, wurde mit der Technologie LaTeX und im Editor VsCode geschrieben.

\subsubsection{Technische Umsetzung} \label{Technische Umsetzung}
Bei der Wahl der Technologien für die technische Umsetzung der Webseite mussten wir sicherstellen, dass die Technologie auf verschiedenen Betriebssystemen läuft, da drei unserer Gruppenmitglieder einen Windows-Laptop und ein Mitglied ein MacBook nutzen.
Wir haben uns entschieden das JavaScript Framework Angular zu verwenden, da es für einen solchen Auftrag geeignet ist und einge der Gruppe bereits mit dieser oder ähnlichen Technologien Erfahrungen gesammelt haben.
Alle Daten, welche auf der Webseite dargestellt werden, befinden sich in verschiedenen JSON-Dateien.

TBD

\subsection{Kontrollieren} \label{Kontrollieren}
TBD

\subsection{Auswerten} \label{Auswerten}
TBD

