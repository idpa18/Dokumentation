\documentclass[../main.tex]{subfiles}
Die Frage bezüglich des Aufbaus eines Reduits können wir nun ganz genau beantworten: Lange und düstere Gänge verbinden die Schiessstände mit dem Unterkunftstrakt. Die Betonwände verleihen der Festung eine gewisse Kälte und lassen dabei den Unterkunftstrakt direkt als wohnlich erscheinen. Wenn man bedenkt, in welch kurzer Zeit diese ganzen Anlagen aus dem Boden «gestampft» wurden, so muss gestanden werden, dass an sehr viele Kleinigkeiten gedacht wurde. So hat beispielsweise die gesamte Festung ein Gefälle nach aussen, damit das Wasser aus der Festung fliesst, und sich nicht in der Festung anstaut.
\paragraph{}Anders sieht es bei der Frage, wie sich die Schweizer Armee denn verhielt, aus. Durch die erarbeiteten Operationsbefehle lässt sich sagen, dass wir uns während der Kriegszeit immer am Ausland angepasst haben, dies ist an den Veränderungen der Truppen und den teilweise vorhandenen Begründungen in den Operationsbefehlen zu entnehmen. Die Schweiz verhielt sich ganz offensichtlich zurückhaltend, hatte aber nicht zuletzt wegen ihrer undankbaren Lage im Zentrum Europas immer einen Plan, wie sie dennoch gewisse Teile der Schweiz im Falle eines gegnerischen Überfalles hätte retten können. 
\paragraph{}Die entstandene Arbeit stimmt mit den definierten Anforderungen aus dem Pflichtenheft überein. Das Produkt besitzt eine Karte, die die einzelnen Operationsbefehle und die Bunker der Schweiz darstellen kann. Die Festung Vitznau ist sehr detailliert ersichtlich, mit einer Karte kann man durch die verschiedenen Räume navigieren, die einzelnen Räume haben dann eine Fotogalerie und einen informativen Text. Die Webseite besitzt auch einen Teil mit den wichtigsten Personen, mit Text und Bild über Henri Guisan, Max Alphons Pfyffer von Altishofen, Oscar Adolf Germann und Samuel Gonard. Beim Aufruf der Webseite erhält man einen Überblick über das Thema, damit sind alle gestellten Anforderungen vorhanden.
Anpassungen im Vergleich zum Pflichtenheft hat es beim Layout der Webseite gegeben. Das Menu wurde vereinfacht, und somit die Benutzerfreundlichkeit erhöht.
\paragraph{}Mit unserem Produkt sind wir zufrieden, weil es unserer Meinung nach einen informativen Mehrwert bietet. Die Besucher können sich die verschiedenen Strategien und Bunker grafisch ansehen, sich über die wichtigen Personen informieren und eine Festung erkunden. Das Feedback von unseren Testpersonen war positiv.
\paragraph{}Die Informations-Phase dauerte etwas länger als wir das ursprünglich geplant hatten. Die Analyse der 22 Operationsbefehlen gestaltete sich als zeitaufwendig. Um später eine Karte zu erstellen mussten verschiedene Eckpunkte auf der Schweizerkarte lokalisiert und eingetragen werden. Dies führte zu Schwierigkeiten, weil die Operationsbefehle mit einer Schreibmaschine verfasst, und später eingescannt wurde. Dies machte das Lesen der Ortschaften schwieriger. Zudem waren gewisse Orte auf Google Maps nicht mehr zu finden, weil es sie nicht mehr gab, oder weil es sich um geografische Punkte und keine Ortschaften handelte. Mit einer Internetrecherche konnten wir die fraglichen Punkte aber ausfindig machen.
\paragraph{}Der grobe Ablauf der IDPA hatten wir bereits vor der Informationsphase geplant. In der Planungsphase vertieften wir die Planung für den Realisierungs-Part. Für die Planung und Arbeitsaufteilung verwendeten wir das Onlinetool clickup.com, mit welchem wir gute Erfahrungen gemacht haben. Das Tool ist simpel, man sieht sofort welche Arbeiten noch erledigt werden müssen und welche Tasks in Verzug sind.
\paragraph{}Die Entscheidungsphase verlief ziemlich kurz. Wir mussten uns nur über den Produktinhalt einigen treffen, was schnell von statten ging. Im Pflichtenheft hatten wir die Funktionalitäten schon beschrieben, was die Entscheidungsfindung erleichterte. Die schwierigste Entscheidung war die Frage nach der Umsetzung der Karten-Funktionalität. Wie bereits im Abschnitt ‘’ 4.4.2 Technische Umsetzung’’ beschrieben, entschieden wir uns dafür ein statisches Bild zu verwenden, welches zwar die Funktionalität des Zoomens nicht bietet, dafür die Umsetzung erleichtert.
\paragraph{}Der Verlauf der Realisierungsphase verlief ohne nennenswerte Probleme. Hierbei ist anzumerken, dass sich zu dieser Zeit unser Projekt auf interkontinentaler Basis abspielte, weil ein Gruppenmitglied einen Auslandaufenthalt in Vietnam machte. Die Zusammenarbeit verlief trotz fünf Stunden Zeitverschiebung tadellos.
Durch den Entscheid, ein Bild für die Karte zu verwenden, hat sich eine grosse Ungewissheit in der Umsetzung eliminiert. Die Realisierung verlief wie geplant, allerdings zeitlich etwas nach hinten verschoben, weil die Informations-Phase länger gedauert hatte.
\paragraph{}Zur Kontrolle unserer Arbeit haben wir die Dokumentation zum Korrekturlesen an mehrere Personen gegeben und deren Korrekturen angewendet. Die Webseite haben wir mehreren Testpersonen zur Verfügung gestellt, und so die Benutzerfreundlichkeit getestet. Das Feedback aus diesen Tests haben wir ebenfalls verarbeitet.
\paragraph{}Zusammenfassend kann man sagen, das während unserer Arbeit keine schwerwiegenden Probleme auftraten, und wir unsere Anforderungen erfüllen konnten. Mit unserem Produkt sind wir zufrieden.