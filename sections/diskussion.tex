\documentclass[../main.tex]{subfiles}
Die Frage bezüglich des Aufbaus eines Reduits können wir nun ganz genau beantworten: Lange und düstere Gänge verbinden die Schiessstände mit dem Unterkunftstrakt. Die Betonwände verleihen der Festung eine gewisse Kälte und lassen dabei den Unterkunftstrakt direkt als wohnlich erscheinen. Wenn man bedenkt, in welch kurzer Zeit diese ganzen Anlagen aus dem Boden «gestampft» wurden, so muss gestanden werden, dass an sehr viele Kleinigkeiten gedacht wurde. So hat beispielsweise die gesamte Festung ein Gefälle nach aussen, damit das Wasser aus der Festung fliesst, und sich nicht in der Festung anstaut.
\paragraph{}Anders sieht es bei der Frage, wie sich die Schweizer Armee denn verhielt, aus. Durch die erarbeiteten Operationsbefehle lässt sich sagen, dass wir uns während der Kriegszeit immer am Ausland angepasst haben, dies ist an den Veränderungen der Truppen und den teilweise vorhandenen Begründungen in den Operationsbefehlen zu entnehmen. Die Schweiz verhielt sich ganz offensichtlich zurückhaltend, hatte aber nicht zuletzt wegen ihrer undankbaren Lage im Zentrum Europas immer einen Plan, wie sie dennoch gewisse Teile der Schweiz im Falle eines gegnerischen Überfalles hätte retten können. 

