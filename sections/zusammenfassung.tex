\documentclass[../main.tex]{subfiles}
\subsection{Fragestellung} \label{Fragestellung}
Im Zusammenhang unserer interdisziplinären Projektarbeit stellten wir uns folgende Fragen:
\begin{itemize}
    \item Wie verhielt sich die Schweizer Armee im zweiten Weltkrieg?
    \item Wie sieht eine Reduit-Festung von innen aus?
 \end{itemize}

 \subsection{Methode} \label{Methode}
 Als Projektmethode haben wir IPERKA (informieren, planen, entscheiden, realisieren, kontrollieren, auswerten). Die Methode kannten wir bereits aus dem Berufsschulunterricht, deshalb bot sie sich an. Der Ablauf der Methode passt gut für eine wissenschaftliche Arbeit. Zuerst informierten wir uns über das Thema, planten dann den Projektablauf, entschieden unsere Inhalte, realisierten die Arbeit und die technische Umsetzung, kontrollierten am Schluss und machten eine Auswertung in Form einer Reflexion.
 \paragraph{}Eine andere Projektmethode, die wir kannten, ist HERMES. Wir haben uns aber gegen HERMES entschieden, einerseits weil die Methode für IT-Projekte und nicht für wissenschaftliche Arbeiten ausgelegt ist, andererseits weil mit HERMES viele Dokumente hätten erstellt werden müssen, die uns in dieser Arbeit nichts gebracht hätten.

 \subsection{Ergebnisse} \label{Ergebnisse}
 Aus unserer Arbeit resultierte einerseits eine schriftliche Dokumentation, andererseits eine Webseite, die die erarbeiteten Inhalte darstellt. Die Webseite kann unter \url{https://eine-wehrhafte-schweiz.kairos-solutions.ch/} aufgerufen werden.

 \subsection{Diskussion} \label{Diskussion}
 TBD - Fragestellungen beantworten
