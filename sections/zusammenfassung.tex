\documentclass[../main.tex]{subfiles}
\subsection{Fragestellung} \label{Fragestellung}
Im Zusammenhang mit unserer interdisziplinären Projektarbeit stellten wir uns folgende Fragen:
\begin{itemize}
    \item Wie verhielt sich die Schweizer Armee im zweiten Weltkrieg?
    \item Welche Räumlichkeiten befinden sich in einer Reduit-Festung?
 \end{itemize}

 \subsection{Methode} \label{Methode}
 Als Projektmethode benutzten wir IPERKA (Informieren, Planen, Entscheiden, Realisieren, Kontrollieren, Auswerten). Die Methode kannten wir bereits aus dem Berufsschulunterricht, deshalb bot sie sich an. Der Ablauf der Methode ist für eine wissenschaftliche Arbeit passend. Zuerst informierten wir uns über das Thema, planten dann den Projektablauf, entschieden über unsere Inhalte, realisierten die Arbeit und die technische Umsetzung, kontrollierten am Schluss sämtliche Ergebnisse und erstellen eine Auswertung in Form einer Reflexion.
 \paragraph{}Eine andere uns bekannte Projektmethode ist HERMES. Wir haben uns aber gegen HERMES entschieden, einerseits ist die Methode für IT-Projekte und nicht für wissenschaftliche Arbeiten ausgelegt, und andererseits beinhaltet die Methodik HERMES das Erstellen einer Vielzahl an Dokumenten, die uns in dieser Arbeit keinen Mehrwert geboten hätten.

 \subsection{Ergebnisse} \label{Ergebnisse}
 Aus unserer Arbeit resultierte als Hauptprojekt eine Webseite, welche die erarbeiteten Inhalte darstellt. Ebenfalls gehört eine Dokumentation zum Ergebnis. Die Webseite kann unter \url{https://eine-wehrhafte-schweiz.kairos-solutions.ch/} aufgerufen werden.

 \subsection{Diskussion} \label{Diskussion}
Die beiden uns gestellten Fragen konnten wir beide beantworten. Die Reduit-Bunker beinhalten Schiessstände, Lagerräume und einen Wohntrakt. Über lange Gänge sind die Räume miteinander verbunden.
Im zweiten Weltkrieg verhielt sich die Schweiz eher passiv, der Fokus lag auf der Verteidigung. Mit insgesamt 22 Operationsbefehlen wurde die Verteidigungsstrategie jeweils den Umständen im Ausland angepasst.
\paragraph{}TBD - Reflexion zusammenfassen
