\documentclass[../main.tex]{subfiles}
\subsection{Fragestellung} \label{Fragestellung}
Im Zusammenhang unserer interdisziplinären Projektarbeit stellten wir uns folgende Fragen:
\begin{itemize}
    \item Wie verhielt sich die Schweizer Armee im zweiten Weltkrieg?
    \item Wie sieht eine Reduit-Festung von innen aus?
 \end{itemize}

 \subsection{Methode} \label{Methode}
 Als Projektmethode haben wir IPERKA (Informieren, Planen, Entscheiden, Realisieren, Kontrollieren, Auswerten). Die Methode kannten wir bereits aus dem Berufsschulunterricht, deshalb bot sie sich an. Der Ablauf der Methode passt gut für eine wissenschaftliche Arbeit. Zuerst informierten wir uns über das Thema, planten dann den Projektablauf, entschieden über unsere Inhalte, realisierten die Arbeit und die technische Umsetzung, kontrollierten am Schluss sämmtliche Ergebnisse und erstellen eine Auswertung in Form einer Reflexion.
 \paragraph{}Eine andere uns bekannte Projektmethode ist HERMES. Wir haben uns aber gegen HERMES entschieden, einerseits ist die Methode für IT-Projekte und nicht für wissenschaftliche Arbeiten ausgelegt, und andererseits beinhaltet die Methodik HERMES das Erstellen einer Vielzahl an Dokumenten, die uns in dieser Arbeit keinen Mehrwert geboten hätten.

 \subsection{Ergebnisse} \label{Ergebnisse}
 Aus unserer Arbeit resultierte als Hauptprojekt eine Webseite, welche die erarbeiteten Inhalte darstellt. Ebenfalls gehört eine Dokumentation zum Ergebnis. Die Webseite kann unter \url{https://eine-wehrhafte-schweiz.kairos-solutions.ch/} aufgerufen werden.

 \subsection{Diskussion} \label{Diskussion}
 TBD - Fragestellungen beantworten
