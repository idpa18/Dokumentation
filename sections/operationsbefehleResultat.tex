\documentclass[../main.tex]{subfiles}
\subsubsection{Operationsbefehl 1} \label{Operationsbefehl 1}
TBD

\subsubsection{Operationsbefehl 2} \label{Operationsbefehl 2}
TBD

\subsubsection{Operationsbefehl 3} \label{Operationsbefehl 3}
TBD

\subsubsection{Operationsbefehl 4} \label{Operationsbefehl 4}
TBD

\subsubsection{Operationsbefehl 5} \label{Operationsbefehl 5}
Der Operationsbefehl 5 wurde am 30. März 1940 ausgestellt. In diesem Operationsbefehl werden die Tätigkeiten des ersten, zweiten und vierten Armeekorps beschrieben, das dritte Armeekorps verändert sich nicht. Die Verteidigung der Grenze wurde wie folgt aufgeteilt:
\begin{itemize}
\item Das erste Armeekorps hält die Festungen St. Maurice im Wallis und St. Gotthard. Dazwischen verhindern sie das Durchstossen der Gegner durch das Wallis auf die Berner Alpen. Weiter sperren sie bei Osogna den Durchgang in die obere Leventina und ins Bleniotal.
\item Das zweite Armeekorps führt mit den Vortruppen einen Verzögerungskampf von der Landesgrenze bis zur Armeestellung, welche die Linie von Basel über Delémont nach Undervelier beinhaltet. Das Gempenplateau wird als rechte Flügelstellung genutzt. Zudem verhindert das zweite Armeekorps alle Vorstösse in Richtung Paswang, Scheltenpass und Moutier, und baut den Hauenstein als hinteren Stützpunkt aus.
\item Das vierte Armeekorps verhindert einen raschen Durchstoss des Feindes aus Splügen-Oberengadin ins Vorderrheintal. Weiter verzögert es einen feindlichen Vorstoss aus dem Unterengadin und hält sich bereit, einem gleichzeitig erfolgenden Angriff aus dem Norden entgegenzutreten.
\end{itemize}

\subsubsection{Operationsbefehl 6} \label{Operationsbefehl 6}
TBD

\subsubsection{Operationsbefehl 7} \label{Operationsbefehl 7}
TBD

\subsubsection{Operationsbefehl 8} \label{Operationsbefehl 8}
TBD

\subsubsection{Operationsbefehl 9} \label{Operationsbefehl 9}
TBD

\subsubsection{Operationsbefehl 10} \label{Operationsbefehl 10}
TBD

\subsubsection{Operationsbefehl 11} \label{Operationsbefehl 11}
TBD

\subsubsection{Operationsbefehl 12} \label{Operationsbefehl 12}
Vom Operationsbefehl 12 gibt es mehrere Versionen. Vom 26.07.1940 ab 06:00 bis am 01.01.1941 war der Operationsbefehl 12 gültig, welcher der Operationsbefehl 11 ablöste. Danach wurde der Operationsbefehl 12 bis. aktiviert. Die beiden Operationsbefehle unterschieden sich nicht wesentlich, in der zweiten Version wurde auf die vorgeschobenen Detachemente, also kleine Truppenabteilungen, welche etwas vor der Hauptverteidigungslinie platziert sind, im Jura verzichtet. Zudem wurde die Verteidigungslinie des vierten Armeekorpses bis Calanda verlängert, ursprünglich war sie nur bis zum Ringelspitz geplant gewesen. Zudem wurde aus den Truppen an der südlichen Grenze das fünfte Armeekorps gegründet. Das Verteidigungsgebiet wurde unter den Armeekorps wie folgt aufgeteilt:
\begin{itemize}
    \item Das erste Armeekorps schützt die westliche Grenze, von Hauteville im Kanton Fribourg nach Laupen im Kanton Bern, sowie von St. Aubin am Neuenburgersee bis zum Mont Raimeux, welcher auf der Grenze zwischen den Kantonen Jura und Bern steht. Das erste Armeekorps erhielt die Anweisung, an Ort und Stelle zu bleiben, auch wenn sie vom Gegner umgangen und von den restlichen Armeetruppen abgeschnitten werden.
    \item Das zweite Armeekorps ist für die nördliche Grenze zwischen dem Mont Raimeux und Thalwil am Zürichsee. Ein vorgeschobenes Detachement sperrt das Engnis zwischen Baden und dem Bözberg. Auch für das zweite Armeekorps gilt, an Ort und Stelle zu bleiben, auch wenn sie vom Gegner umlaufen werden.
    \item Das dritte Armeekorps wird als zweite Front Richtung Nordwesten zwischen der Kaiseregg und dem Pilatus installiert.
    \item Das vierte Armeekorps überwacht die Grenze zwischen dem Zürichsee und dem Vierwaldstättersee. Zudem schützt es die Linie zwischen der Rigi über Tödi bis Calanda, beziehungsweise der Ringelspitz in der ersten Version.
    \item Das neu gegründete, fünfte Armeekorps ist für den Schutz in südliche Richtung zuständig. Es schützt das Land vom Tête Blanche durch das Wallis und Tessin bis in den Kanton Graubünden. Zudem schützt das fünfte Armeekorps die Linie von Gletsch über den Gotthard, Airolo und Oberalstock nach Tödi im Kanton Glarus.
\end{itemize}

\subsubsection{Operationsbefehl 13} \label{Operationsbefehl 13}
TBD

\subsubsection{Operationsbefehl 14} \label{Operationsbefehl 14}
Der Operationsbefehl 14 wurde im Jahre 1942 ausgestellt. Im Vergleich zum Operationsbefehl 13 ist die Armee deutlich offensiver aufgestellt, sie befindet sich nicht mehr im Alpen-Reduit. Das dritte Armeekorps veränderte sich im Vergleich zum Operationsbefehl 13 nicht. Die anderen Korps fassten folgende Aufgaben:
\begin{itemize}
    \item Das erste Armeekorps schützt die Grenzen im Westen der Schweiz, von Biaufond bis zum Genfersee und vom Genfersee bis zum Tête Blanche.
    \item Das zweite Armeekorps schützt die Grenze zwischen Biaufond über Basel zum Aargauischen Mumpf. In einer zweiten Linie schützt das zweite Armeekorps das Gebiet zwischen dem Bielersee und Olten, mit der Territorialdivision wird die Reduitstellung zwischen der Rigi und dem Hohgant geschützt. Weiter hat das zweite Armeekorps die Aufgabe, die Brünigbahnlinie zu schützen.
    \item Das dritte Armeekorps schützt zusätzlich die Gotthardbahnlinie und die Lötschberg-Südrampe.
    \item Das vierte Armeekorps schützt die Gotthardbahnlinie mit dem Flugabwehr-Truppen und verteidigt das Gebiet zwischen Mumpf und Sargans.
    \item Die Flugabwehrtruppen schützt zuerst die Mobilmachung und der Aufmarsch eigener Truppen vor feindlichen Angriffen aus der Luft. Danach werden sie dem ersten und dem zweiten Armeekorps unterstellt, zum Schutz der Kampf-Truppen und der Armee-Reserven, sowie für den Kampf gegen feindliche Luftlandetruppen im Mittelland.
    \item Die Flieger sichern die Mobilmachung und Aufmarsch vor Angriffen aus der Luft, unterstützen danach die Erd-Truppen gegen Luft- und Panzerangriffe.
\end{itemize}

\subsubsection{Operationsbefehl 15} \label{Operationsbefehl 15}
Der Operationsbefehl Nummer 15 wurde am 05. Januar 1944 ausgestellt. Im Vergleich zum Operationsbefehl Nummer 14 ist die Strategie defensiver. Die Armee verlässt die Grenzen und zieht sich in den sogenannten Armeeraum zurück. Der Armeeraum wurde in zwei Fronten unterteilt: Die Nordfront von Büren bis Flums und die Südwestfront von Büren bis Guttannen im Kanton Bern. Zusätzlich werden Truppen in den Alpen zurückgelassen. Die Verteidigung sah die folgende Aufgabenverteilung vor:
\begin{itemize}
    \item Das erste Armeekorps riegelt die wichtigsten Eingänge in den Zentralraum vom Grünenberg (Melchnau) bis La Teine ab. Zudem baut es das Kandertal von Mitholz bis zum Lötschberg-Nordportal zu einem geschlossenen Kampfraum aus und besetzt es. Das erste Armeekorps ist für die Verteidigung der südwestlichen Front von Büren bis Guttannen.
    \item Das zweite Armeekorps schützt die Nordwestliche Verteidigungslinie vpn Büren bis zum Hallwilersee. Weiter riegelt das zweite Armeekorps die Lopperstrase ab, besetzt den Brünigpass und ist für die Sicherung der Räume von Stans und Sarnen zuständig.
    \item Das dritte Armeekorps schützt den südöstlichen Armeeraum von Flums bis Guttannen.
    \item Das vierte Armeekorps ist für die Verteidigung des nordöstlichen Armeeraums von Flums bis zum Hallwilersee zuständig. Es riegelt die wichtigsten Eingänge in den Zentralraum vom Klausenpass bis Vitznau zur Sicherung des Talkessels von Schwyz ab.
\end{itemize}

\subsubsection{Operationsbefehl 16} \label{Operationsbefehl 16}
TBD

\subsubsection{Operationsbefehl 17} \label{Operationsbefehl 17}
TBD

\subsubsection{Operationsbefehl 18} \label{Operationsbefehl 18}
TBD

\subsubsection{Operationsbefehl 19} \label{Operationsbefehl 19}
TBD

\subsubsection{Operationsbefehl 20} \label{Operationsbefehl 20}
TBD

\subsubsection{Operationsbefehl 21} \label{Operationsbefehl 21}
TBD

\subsubsection{Operationsbefehl 22} \label{Operationsbefehl 22}
TBD