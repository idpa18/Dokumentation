\documentclass[../main.tex]{subfiles}
\subsubsection{Operationsbefehl 1} \label{Operationsbefehl 1}
Der erste Operationsbefehl wurde am 01. September 1939, also ein Tag nach Kriegsausbruch ausgestellt.  In diesem Operationsbefehl werden die Truppen an der nördlichen und westlichen Grenze platziert:
\begin{itemize}
    \item Das erste Armeekorps wird an der Grenze von Saint Brais bis zum Genfersee platziert.
    \item Das elfte Gebirgs-Brigate ist in den beiden Pässen vom Simplon und Grimsel platziert.
    \item Die neunte Division beschützt die Gotthardfestung und stellt sicher, dass der Gegner nicht durch Bellinzona in die Schweiz einfallen kann.
\end{itemize}

\subsubsection{Operationsbefehl 2} \label{Operationsbefehl 2}
Am 04. Oktober 1939 wurde der zweite Operationsbefehl ausgestellt. Im Vergleich zum ersten Operationsbefehl wird die Grenze zu Deutschland und Österreich stärker bewacht, während an der westlichen Grenze Truppen abgezogen werden.
\begin{itemize}
    \item Das erste Armeekorps wird an der östlichen Grenze zwischen Wil über Thal und Sargans nach Graubünden platziert.
    \item Das zweite Armeekorps stellt sich zwischen dem Gempenplateau und Lauffohr auf.
    \item Zwischen Lauffohr und dem Zürichsee platziert sich das dritte Armeekorps.
    \item Die neunte Division ist für die westliche Grenze zuständig. Dafür platzieren sie sich im Simplon- und Grimselpass, besetzen die Gotthardfestung und beschützen Bellinzona.
\end{itemize}

\subsubsection{Operationsbefehl 3} \label{Operationsbefehl 3}
Der dritte Operationsbefehl von Henri Guisan im zweiten Weltkrieg wurde am 28. November 1939 ausgestellt. Im Vergleich zum zweiten Operationsbefehl bleibt die Verteidigung im Norden gleich. Das erste Armeekorps wird jedoch vom Osten in den Westen verschoben.
\begin{itemize}
    \item Das erste Armeekorps beschützt die Grenze nach Frankreich von den Franches Montagnes bis zum Genfersee. Zudem stellt sich das erste Armeekorps an der südlichen Grenze von St. Gingolph bis zum Griespass auf, verwenden dabei die Festung Saint Maurice.
    \item Die Stellungen des zweiten und dritten Armeekorps sowie der neunten Division bleiben unverändert.
\end{itemize}

\subsubsection{Operationsbefehl 4} \label{Operationsbefehl 4}
Im vierten Operationsbefehl wird eine Armeestellung definiert. Diese zieht sich von Sargans über den Zürichsee, den Bözberg bis zum Gempenplateau. Einzelne Teile der Armeestellung wurde an die Armeekorps verteilt.
\begin{itemize}
    \item Das erste Armeekorps ist für die südliche Grenze zuständig. Dafür verwenden sie die Festungen Saint Maurice und Gotthard.
    \item Das zweite Armeekorps übernimmt die Verteidigung von Liestal bis Lauffohr.
    \item Für den Bereich der Armeestellung zwischen Lauffohr und Bendlikon ist das dritte Armeekorps zuständig.
    \item Das vierte Armeekorps ist zwischen Bendlikon und Sargans patziert.
    \item Die Division Gempen deckt den Bereich zwischen Pratteln und Basel.
\end{itemize}

\subsubsection{Operationsbefehl 5} \label{Operationsbefehl 5}
Der Operationsbefehl 5 wurde am 30. März 1940 ausgestellt. Darin werden die Tätigkeiten des ersten, zweiten und vierten Armeekorps beschrieben, das dritte Armeekorps verändert sich nicht. Die Verteidigung der Grenze wurde wie folgt aufgeteilt:
\begin{itemize}
\item Das erste Armeekorps hält die Festungen St. Maurice im Wallis und St. Gotthard. Dazwischen verhindern sie das Durchstossen der Gegner durch das Wallis auf die Berner Alpen. Weiter sperren sie bei Osogna den Durchgang in die obere Leventina und ins Bleniotal.
\item Das zweite Armeekorps führt mit den Vortruppen einen Verzögerungskampf von der Landesgrenze bis zur Armeestellung, welche die Linie von Basel über Delémont nach Undervelier beinhaltet. Das Gempenplateau wird als rechte Flügelstellung genutzt. Zudem verhindert das zweite Armeekorps alle Vorstösse in Richtung Paswang, Scheltenpass und Moutierund baut den Hauenstein als hinteren Stützpunkt aus.
\item Das vierte Armeekorps verhindert einen raschen Durchstoss des Feindes aus Splügen-Oberengadin ins Vorderrheintal. Weiter verzögert es einen feindlichen Vorstoss aus dem Unterengadin und hält sich bereit, einem gleichzeitig erfolgenden Angriff aus dem Norden entgegenzutreten.
\end{itemize}

\subsubsection{Operationsbefehl 6} \label{Operationsbefehl 6}
Der Operationsbefehl 6 wurde am 9. März 1940 ausgestellt. Darin werden die Tätigkeiten des ersten, zweiten, dritten und vierten Armeekorps beschrieben, die Grenztruppen werden kaum verändert. Die Verteidigung wurde wie folgt aufgeteilt:
\begin{itemize}
    \item Das erste Armeekorps hält die Stellung entlang der Linie Mentue - Paudéze und Villeneuve - Mont Dolent. Sie verhindern einen Vorstoss in Richtung Moudon und Broyetal, ist bereit einen Angriff an der schweizerisch-italienischen Grenze abzuschlagen und Überwacht das Seeufer von Paudex bis Villeneuve.
    \item Das zweite Armeekorps hält die Stellung entlang der Linie Basel - Birslauf - Grellingen - Büsserach - Fringeli - Delémont - Undervelier. Sie verhindern die Besitznahme des Gempenplateau und den Vorstoss in Richtung Passwang, Scheltenpass und Moutier-Court.
    \item Das dritte Armeekorps hält die Stellung entlang der Linie La Joux - Montagne du Droit - Mont d'Amin - Mont Racinge - Creux du Van - St. Aubin. Sie verhindern einen Vorstoss ins Aaretal auf Lengnau - Biel und gegen Zihlkanal. Sie besetzen dazu Jolimont als Stützpunkt.
    \item Das vierte Armeekorps hält die Stellung der Grenztruppen und der Festung St. Gotthard. Sie bereiten sich auch auf einen Angriff im Abschnitt Tête Blanche - Unterengadin vor.
    \item Die Flugabwehrtruppen schützen die Mobilmachung und der Aufmarsch eigener Truppen vor feindlichen Angriffen aus der Luft.
\end{itemize}

\subsubsection{Operationsbefehl 7} \label{Operationsbefehl 7}
Der Operationsbefehl 7 wurde am 12. April 1940 ausgestellt. Darin werden die Tätigkeiten des ersten und vierten Armeekorps beschrieben. Er diente vor allem der Unterstützung der Aufstellung des Operationsbefehls 6 an der Südseite. Die Verteidigung wurde wie folgt aufgeteilt:
\begin{itemize}
    \item Das erste Armeekorps verhindert einen Vorstoss durchs Unterwallis auf die Berner Alpen und breitet sich auf einen Angriff von Westen und Süden vor.
    \item Das vierte Armeekorps hält die Stellung entlang der Linie Maggiatal - Lago Maggiore - Monte Ceneri - Joriopass, sperrt den Zugang zum Grimsel, hält die Festung St. Gotthard und Sargans, verhindert einen Vorstoss durchs Oberwallis auf die Berner Alpen und Soerrt bei Osogna den Durchgang in Leventina und Bleniotal. Ausserdem verhindern sie einen Vorstoss durch Splügen - Oberengadin ins Vorderreihntal und verzögern einen allfälligen Vorstoss aus dem Unterengadin.
\end{itemize}

\subsubsection{Operationsbefehl 8} \label{Operationsbefehl 8}
Der Operationsbefehl 8 wurde am 22. April 1940 ausgestellt. Darin werden die Tätigkeiten des ersten, zweiten, dritten und vierten Armeekorps sowie der Südfront und der Flugabwehrtruppen beschrieben. Die Verteidigung wurde wie folgt aufgeteilt:
\begin{itemize}
    \item Das erste Armeekorps hält die Stellung entlang der Linie Berner Jura und Mentue - Neuenburger Jura und Mentue - Paudèze, sodass der Feind nicht durch die Linie Moron - Biel - Jolimont - Mentue - Paudèze kommt. Anschliessend soll die Nordtruppe auf die Achse Bern - Thun und die Südtruppe auf Jaun und Château d’Oex wechseln.
    \item Das zweite Armeekorps hält die Stellung entlang der Linie Basel - Birslauf - Grellingen - Gilgenberg - Hohe Winde - Matzendorf - Wiedlisbach - Langenthal.
    \item Das dritte Armeekorps hält die Stellung entlang der Linie Lotzwil - Madiswil - Huttwil - Eriswil - Hohmatt - Trubschachen - Schangnau.
    \item Das vierte Armeekorps hält die Stellung entlang der Linie Honegg - Zulg - Stockhorn - Gantrisch - Kaiseregg - Vanil Noir - Cape au Moine - Chillon - St. Maurice - Tour Sallière - Mont Dolent. Sollte es zwischen Vanil Noir – Chillon Probleme geben, dann nimmt der linke Flügel Stellung entlang der Linie Valin Noir – Diablerets – Dent du Midi – Mont Dolent.
    \item Die Südfront des vierten Armeekorps schützt Mont Dolent bis Gries Pass, die Südfront der 9. Division verstärkt Gries Pass bis Festung Sargans.
    \item Die Flugabwehrtruppen decken den Aufmarsch der Armee und unterstützen das ersten Armeekorps.
\end{itemize}

\subsubsection{Operationsbefehl 9} \label{Operationsbefehl 9}
Der Operationsbefehl 9 wurde am 22. April 1940 ausgestellt. Darin werden die Tätigkeiten des vierten Armeekorps und der 9. Division beschrieben. Die Verteidigung wurde wie folgt aufgeteilt:
\begin{itemize}
    \item Das vierte Armeekorps verweht einen allfälligen Durchstoss durch das Wallis auf die Berner Alpen, sperrt den Zugang zur Grimsel und ist bereit für einen Angriff des Feindes von Westen und Süden.
    \item Die 9. Division hält die Festung St. Gotthard, verzögert einen Vorstoss auf der Linie Maggiatal - Lago Maggiore - Monte Ceneri - Joriopass und sperrt bei Osogna den Durchgang zu Leventina und Bleniotal. Ausserdem verhindern sie einen allfälligen Durchstoss aus Splügen - Oberengadin ins Vorderrheintal und verzögern einen Vorstoss aus Unterengadin.
\end{itemize}
Aufgabe der Deckungstruppen ist es, Zeit zu gewinnen, damit sich die Kräfte an den anderen Fronten umgruppieren können. Diese Umgruppierung kann mehrere Wochen dauern.
Am 5. Juni 1940 stellte Guisan einen Brief aus, in dem er die Umgruppierung der militärischen Kräfte beschrieb, dies da ihm der Durchbruch der belgischen Verteidigungsfront Grund zur Sorge gab. Hier das Wichtigste aus dem Brief:
\begin{itemize}
    \item Die Verteidigung soll an jenen Stellen ausgebaut werden, an denen der Durchbruch für Panzer am einfachsten ist.
    \item An diese Stellen sollen viele Ressourcen stationiert werden, um einen Durchbruch schnellstmöglich aufzuhalten.
    \item Sollten es feindliche Panzer durch die Verteidigungsfront schaffen, soll sie mit allen Mitteln aufgehalten werden. Dies beinhaltet Sperrung oder Sprengung von Strassen und Brücken oder einen Nahkampf mit Sturmtruppen. Die Priorität lag darin, den Panzer lahm zu legen, er soll nicht umkehren und seine Kameraden warnen können.
    \item Die Artillerie soll ebenfalls am Kampf teilnehmen.
    \item Die Umgruppierung hat höchste Priorität.
\end{itemize}
Am 19. Juni 1940 stellte Guisan einen weiteren Brief aus, in dem er eine Verdunkelung des Landes beschrieb, die im Falle einer Invasion ausgeführt werden solle. Bei einer Verdunkelung müssen die Betroffenen Orte jegliches Licht vermeiden, um nicht von feindlichen Fliegern entdeckt zu werden. Hier die wichtigsten Punkte:
\begin{itemize}
    \item Die Verdunkelung werde von Herrn Guisan persönlich ausgerufen und durch Radio und Presse ans Volk kommuniziert.
    \item Wird ein Armeekorps, eine Grenzbrigade oder ein Stadtkommando angegriffen, entscheidet das Armeekommando, ob die Siedlungen in der Umgebung verdunkelt werden sollen.
    \item Eine Verdunkelung kann mehrere Stunden anhalten und sollte sie in der Nacht ausgerufen werden, muss sie umgehend ausgeführt werden.
    \item In Luftschutzpflichten Ortschaften kann ein Armeekorps, eine Division oder eine Grenzbrigade die Verdunkelung einer Stadt selbst berufen.
    \item Sollte ein Fliegeralarm in der Nacht erklingen, soll eine Verdunkelung sofort ausgeführt werden.
    \item Die Kommunikation zwischen dem Kommandanten und zivilen Stellen ist durch die Luftschutzoffiziere gewährleistet.
\end{itemize}

\subsubsection{Operationsbefehl 10} \label{Operationsbefehl 10}
Der Operationsbefehl 10 sollte nach seiner Ausstellung die Operationsbefehle 4, 5, 6, 7, 8 und 9 ersetzen, er hat jedoch kein Datum, an dem er in Kraft tritt und wurde auch nie von Guisan unterschrieben. Darin werden die Tätigkeiten des ersten, zweiten, dritten und vierten Armeekorps sowie der 9. Division, der Flugabwehrtruppen und der Überwachungstruppen beschrieben. Die Verteidigung wurde wie folgt aufgeteilt:
\begin{itemize}
    \item Das erste Armeekorps hält die Stellung bei Délemont zwischen Wasserberg und Les Rangiers gegen Norden. Es sperrt ebenfalls zwischen Courchapoix und Glovelier alle Wege nach Osten, Süden und Südwesten, deckt Franches Monatges durch Doubseinschnitt und hält die Stellung entlang der Linie Mentue - Jorat - Paudéze und die Festung St. Maurice.
    \item Das zweite Armeekorps deckt die Linie Olten - Balsthal gegen Norden, hält die Front Lauffohr - Geissberg - Frickberg - Thiersteinberg - Farnsberg - Gempen und verteidigt auf der Länge Aare von Koblenz bis Brugg mit dem Stützpunkt beim Bözberg. Ebenfalls hält es Hauenstein und Waldenburg, deckt Passwang und Schelten gegen Norden und Westen und sperrt Dürrental bei Welschenrohr.
    \item Das dritte Armeekorps verteidigt auf Länge Limmat mit dem Stützpunkt bei Mutschellen.
    \item Das vierte Armeekorps hält den Abschnitt zwischen Festung Sargans und Thalwil gegen Osten und Nordosten, verhindert ein Vordringen in Richtung Glarus, Schwyz und Zug und hält die Linie Somvix - Landquart.
    \item Die 9. Division deckt das Berner Oberland, hält den Gotthard gegen Westen, Süden und Osten und verhindert im Wallis einen Vorstoss ins Rhonetal mit dem Schwergewicht auf den Simplon. Ausserdem hält es die Linie Valle Maggia - Magadino - Monte Ceneri - Passo di Jorio und stellt in Graubünden am Greina- und Disrutpass die Verbindung zum vierten Armeekorps sicher.
    \item Die die Flieger der Flugabwehrtruppen sichern die Truppen, welche vor dem dritten und vierten Armeekorps Brücken und Strassen zerstören und die Abwehrtruppen decken das Aaretal zwischen Brugg und Biel, das Reusstal zwischen Brugg und Luzern, Birstal bei Laufen und wichtige Flugplätze.
    \item Überwachungstruppen bekämpfen mit den Ortswehren feindliche Fallschirmspringer, Saboteure und eingedrungene Fahrzeuge.
\end{itemize}
\subsubsection{Operationsbefehl 11} \label{Operationsbefehl 11}
Der Operationsbefehl 11 wurde am 12. Juli 1940 ausgestellt. Darin werden die Tätigkeiten der ersten, dritten, siebten und achten Division beschrieben. Die Verteidigung wurde wie folgt aufgeteilt:
\begin{itemize}
    \item Die erste Division hält die Stellung entlang der Linie La Berra - Sorens - Vaulruz - Moléson - Rochers de Naye - Gumfluh - Saanen - Jaunpass.
    \item Die dritte Division hält die Stellung entlang der Linie Zulg - Guggisberg - Lac Noir - Boltigen - Gsür - Niesen - Scheibe - Eriz.
    \item Die siebte Division hält die Stellung entlang der Linie Weesen - Grynau - Wädenswil - Cham - Küssnacht - Gersau - Steinen - Gr. Mythen - Drusberg - Scheye.
    \item Die achte Division hält die Stellung entlang der Linie Luzern - Malters - Wohlhusen - Schüpfheim - Flühli - Brünig - Melchsee - Jochpass - Stans - Bouchs.
\end{itemize}

\subsubsection{Operationsbefehl 12} \label{Operationsbefehl 12}
Vom Operationsbefehl 12 erschienen mehrere Versionen. Er löste den Operationsbefehl 11 ab und war vom 26.07.1940 ab 06:00 bis am 01.01.1941 gültig. Danach wurde der Operationsbefehl 12 bis. aktiviert. Die beiden Operationsbefehle unterschieden sich nicht wesentlich, in der zweiten Version wurde auf die vorgeschobenen Detachemente, also kleine Truppenabteilungen, welche etwas vor der Hauptverteidigungslinie platziert sind, im Jura verzichtet. Zudem wurde die Verteidigungslinie des vierten Armeekorps bis Calanda verlängert, ursprünglich war sie nur bis zum Ringelspitz geplant gewesen. Ausserdem wurde aus den Truppen an der südlichen Grenze das fünfte Armeekorps gegründet. Das Verteidigungsgebiet wurde unter den Armeekorps wie folgt aufgeteilt:
\begin{itemize}
    \item Das erste Armeekorps schützt die westliche Grenze von Hauteville im Kanton Fribourg nach Laupen im Kanton Bern sowie von St. Aubin am Neuenburgersee bis zum Mont Raimeux, welcher auf der Grenze zwischen den Kantonen Jura und Bern steht. Das erste Armeekorps erhielt die Anweisung, an Ort und Stelle zu bleiben, auch wenn sie vom Gegner umgangen und von den restlichen Armeetruppen abgeschnitten werden.
    \item Das zweite Armeekorps ist für die nördliche Grenze zwischen dem Mont Raimeux und Thalwil am Zürichsee. Ein vorgeschobenes Detachement sperrt den Engpass zwischen Baden und dem Bözberg. Auch für das zweite Armeekorps gilt, an Ort und Stelle zu bleiben, auch wenn sie vom Gegner umlaufen werden.
    \item Das dritte Armeekorps wird als zweite Front Richtung Nordwesten zwischen der Kaiseregg und dem Pilatus installiert.
    \item Das vierte Armeekorps überwacht die Grenze zwischen dem Zürichsee und dem Vierwaldstättersee. Zudem schützt es die Linie zwischen der Rigi über Tödi bis Calanda beziehungsweise der Ringelspitz in der ersten Version.
    \item Das neu gegründete fünfte Armeekorps ist für den Schutz in südliche Richtung zuständig. Es schützt das Land vom Tête Blanche durch das Wallis und Tessin bis in den Kanton Graubünden. Zudem schützt es die Linie von Gletsch über den Gotthard, Airolo und Oberalpstock nach Tödi im Kanton Glarus.
\end{itemize}

\subsubsection{Operationsbefehl 13} \label{Operationsbefehl 13}
Der am 22. April 1941 in einer ersten Version erstellte Operationsbefehl Nr. 13 hatte während den Kriegsjahren mehrere verschiedene Nachfolgeversionen. Diese unterschieden sich im Wesentlichen nicht voneinander. Als einzige Änderungen sind Truppenzugehörigkeiten der Armeekorps aufzufinden. Der Operationsbefehl 13 verfolgte eine zurückgezogene Stellung, in der die Grenzstädte Basel und Genf dem Gegner ohne jeglichen Kampf überlassen worden wären. Die nördliche Verteidigungslinie lag erst im Alpenraum. Diese Zurückhaltung wurde Zentralraumstellung genannt. Der Frontverlauf wurde durch die 5 Armeekorps wie folgt verteidigt: 
\begin{itemize}
    \item Das erste Armeekorps war für den westlichen Abschnitt zwischen Kaiseregg und Tête Blanche verantwortlich. Zwischen den beiden Eckpunkten galt es besonders die Waadtländer Alpen sowie alle Pässe zwischen Muveran und Wildstrubel zu verteidigen.
    \item Von der Kaiseregg weiter westwärts bis zum Hohgant besass das zweite Armeekorps ihr Revier. Hier galt besonders das Simmen- und Kandertal zu besetzten, damit das Berner Oberland gedeckt war. 
    \item Noch westlicher war das dritte Armeekorps beauftragt vom Rigi über den Brünig bis zum Hohgant die Zentralraumstellung zu schützen. 
    \item Wobei der Rigi in den Bereich des vierten Armeekorps gehörte: vom Rigi bis zum Tödi, wobei die Linie eine westlich ausgeprägte Kurve bis nach Sargans darstellt. So wurde das südliche Zürichseeufer geschützt. 
    \item Die südliche Front wurde vom fünften Armeekorps verteidigt und reichte vom Tödi bis zum Tête Blanch.
\end{itemize}


\subsubsection{Operationsbefehl 14} \label{Operationsbefehl 14}
Der Operationsbefehl 14 wurde im Jahre 1942 ausgestellt. Im Vergleich zum Operationsbefehl 13 ist die Armee deutlich offensiver aufgestellt, sie befindet sich nicht mehr im Alpen-Reduit. Das dritte Armeekorps veränderte sich im Vergleich zum Operationsbefehl 13 nicht. Die anderen Korps fassten folgende Aufgaben:
\begin{itemize}
    \item Das erste Armeekorps schützt die Grenzen im Westen der Schweiz, von Biaufond bis zum Genfersee und vom Genfersee bis zum Tête Blanche.
    \item Das zweite Armeekorps schützt die Grenze zwischen Biaufond über Basel zum aargauischen Mumpf. In einer zweiten Linie schützt das zweite Armeekorps das Gebiet zwischen dem Bielersee und Olten, mit der Territorialdivision wird die Reduitstellung zwischen der Rigi und dem Hohgant geschützt. Weiter hat das zweite Armeekorps die Aufgabe, die Brünigbahnlinie zu schützen.
    \item Das dritte Armeekorps schützt zusätzlich die Gotthardbahnlinie und die Lötschberg-Südrampe.
    \item Das vierte Armeekorps schützt die Gotthardbahnlinie mit den Flugabwehr-Truppen und verteidigt das Gebiet zwischen Mumpf und Sargans.
    \item Die Flugabwehrtruppen schützen zuerst die Mobilmachung und der Aufmarsch eigener Truppen vor feindlichen Angriffen aus der Luft. Danach werden sie dem ersten und dem zweiten Armeekorps unterstellt, zum Schutz der Kampf-Truppen und der Armee-Reserven sowie für den Kampf gegen feindliche Luftlandetruppen im Mittelland.
    \item Die Flieger sichern die Mobilmachung und Aufmarsch vor Angriffen aus der Luft, unterstützen danach die Erdtruppen gegen Luft- und Panzerangriffe.
\end{itemize}

\subsubsection{Operationsbefehl 15} \label{Operationsbefehl 15}
Der Operationsbefehl Nummer 15 wurde am 05. Januar 1944 ausgestellt. Im Vergleich zum Operationsbefehl Nummer 14 ist die Strategie defensiver. Die Armee verlässt die Grenzen und zieht sich in den sogenannten Armeeraum zurück. Dieser wurde in zwei Fronten unterteilt: die Nordfront von Büren bis Flums und die Südwestfront von Büren bis Guttannen im Kanton Bern. Zusätzlich werden Truppen in den Alpen zurückgelassen. Die Verteidigung sah die folgende Aufgabenverteilung vor:
\begin{itemize}
    \item Das erste Armeekorps riegelt die wichtigsten Eingänge in den Zentralraum vom Grünenberg (Melchnau) bis La Teine ab. Zudem baut es das Kandertal von Mitholz bis zum Lötschberg-Nordportal zu einem geschlossenen Kampfraum aus und besetzt es. Das erste Armeekorps ist für die Verteidigung der südwestlichen Front von Büren bis Guttannen.
    \item Das zweite Armeekorps schützt die nordwestliche Verteidigungslinie von Büren bis zum Hallwilersee. Weiter riegelt das zweite Armeekorps die Lopperstrasse ab, besetzt den Brünigpass und ist für die Sicherung der Räume von Stans und Sarnen zuständig.
    \item Das dritte Armeekorps schützt den südöstlichen Armeeraum von Flums bis Guttannen.
    \item Das vierte Armeekorps ist für die Verteidigung des nordöstlichen Armeeraums von Flums bis zum Hallwilersee zuständig. Es riegelt die wichtigsten Eingänge in den Zentralraum vom Klausenpass bis Vitznau zur Sicherung des Talkessels von Schwyz ab.
\end{itemize}

\subsubsection{Operationsbefehl 16} \label{Operationsbefehl 16}
Der Operationsbefehl 16 wurde am 01. Januar 1944 ausgestellt und tritt auf Guisans Befehl hin in Kraft. Darin werden die Tätigkeiten des erste, zweiten, dritten und vierten Armeekorps beschrieben. Die Verteidungung wurde wie folgt aufgeteilt:
\begin{itemize}
    \item Das erste Armeekorps hält dass Unterwallis und deckt die rechte Flanke auf den Jurakamm von Mont Aubert bis Mont de Biere mit Anschluss an Promenthouse und Aubonne. Es sperrt ebenfalls alle Eingänge zum Zentralraum und hält die Stellung entlang der Linie Klösterli – Liesberg – Spitzbühl - Pt. 949 – Chenal – Corba – Montaigo – Schönenberg – Karlisberg – Probstenberg – Welschenrohr – Balmberg – Gallmoos – Hofmatt – Solothurn – Biberist – Löffelberg – Kyburg – Aettingen – Limpach – Etzelkofen – Iffwil – Zuzwil – Münchenbuchsee – Hubel – Herrenschwanden – Bümplis – Könz . Ob. Ulmiz – Zimmerwald – Niedermuhlern – Pt. 960 – Hasli – Riggisberg – Schönegg – Wattenwil – Weiermoos – Langenegg – Hohmaad – Schwiedeneggrat – Oberwil – Schwendi – Niederhorn – Geisfluh – Seeberg – Frohmattgrat – Spillgerten – Mieschfluh – Albristhorn – Schatthorn – Bühlberg – Stalden – Wildstrubel – Nisey – Salgesch – Pfyn – Bella Tola – Weisshorn – Tete Blanche.
    \item Das zweite Armeekorps überwacht an der Nordfront die Grenze zur Grenzbrigarde 4 - 8 und sperrt alle Eingänge zum Zentralraum zwischen Wallensee und Stochhorn. Es schützt auch die Lufträume und hält die Stellung entlang der Linie Lienz – Hohor Kasten – Alpsiegel – Altenalp – Säntis – Stoss – Stein – Leist – Quinten -Sexmoor – Gugnen – Spitzmeilen – Weissgandstöckli – Foostook – P. Sardona – P. Atlas – A. Nagiens – Sargenserfurka – A-Ranasoa – Carp. Surscheins – Kistenpass – Tödi – Sandalpass – Claridenstock – Grosse Windgälle – Silenen – Grosser Spannort – Titlis – Mühlestalden – Innertkirchen – Wellhorn – Wetterhörner – Jungfrau – Blüemlisalp – Kandersteg – Engstligenalp – Wildstrubel – Stalden – Bühlberg – Schatthorn – Albristhorn – Mieschfluh – Spillgerten – Frohmattgrat – Seeberg – Geisfluh – Niederhorn – Schwendi – Oberwil – Schwiedeneggrat – Hohmaad – Langenegg – Weiermoos – Wattenwil – Schönegg – Riggisberg – Hasli – Pt. 960 – Niedermuhlern – Zimmerwald – Ob. Ulmiz – Köniz – Bümpliz – Herrenschwanden – Hubel – Münchenbuchsee – Zuzwil – Iffwil – Etzolkofen – Limpach – Aettingen – Kyburg – Löffelberg – Biberist – Welchenrohr – Probstenberg – Karlisberg – Schönenberg – Nontaigu – Corban – Chenal – Pt.949 – Spitzenbühl – Liesberg – Klösterli.
    \item Das dritte Armeekorps hält das Oberwallis und Tessin und verhindert einen Druchbruch durch Rheinwald, Unter- und Oberengadin in Richtung des Vorderrheintal. Zudem ist es bereit auf die Linie Pzo. Bianco – Pzo. Tignaga – Cma. Di Capezzone – Cma. Della Grotta – Ornavasso – Pallanza zu wechseln und hält bis dahin die Stellung entlang der Linie L. di Cama – Sorte – Pzo. Di Groveno – Fil. Di Dragiva – Cma. Di Trescolmine – L. di Pasetti – Pzo. Rotondo – Pzo di Muccia – Rheinquellhorn – Rheinwaldhorn – Läntagletscher – A. Lampertsch – P. Terri – Pt. 2472 – P. Cavel – P. Greina – Somvitg – P. Ner – Tödi – Sandalpass – Claridenstock – Grosse Windgälle – Silenen – Grosser Spannort – Titlis – Mühlestalden – Innertkirchen – Wellhorn – Wetterhörner – Jungfrau – Blüemlisalp – Kandersteg – Engstligenalp – Wildstrubel – Nisey – Sagesch – Pfyn – Bella Tola – Weisshorn – Tete Blanche.
    \item Das vierte Armeekorps hält Graubünden und die Festung Sargans und verhindert einen Druchbruch durch Rheinwald, Unter- und Oberengadin in Richgunt Vorderrheintal. Es isch auch bereit auf die Linie M. Martello – Pzo. Rabbi – M. Sasso Canale – Novate – M. Gajazzo – Pzo. Porcellizzo zu wechseln und hält die Stellung entlang der Linie Liens – Hoher Kasten – Alpsiegel – Säntis – Stoss – Stein – Leist – Quinten – Sexmoor – Gugnen – Spitzmeilen – Weissgandstöckli – Foostock – P. Sardona – P. Atlas – A. Nagiens – Sagenserfurka – A. Ranasca – Crap Surscheins – Kistenpass – Tödi – P. Ner – Somvix – P. Grein – P. Cavel – Pt. 2472 – T. Terri – A. Lampertsch – Lenta Gletscher – Rheinwaldhorn – Rheinquellhorn – Pzo. Di Muccia – Pzo. Rotondo – L. di Passetti – Cma. Di Trescolmine – Fil. Di Dragiva – Pzo. Di Groveno – Sorte – L. di Cana.
\end{itemize}

\subsubsection{Operationsbefehl 17} \label{Operationsbefehl 17}
Der Operationsbefehl 17 wurde am 01. Januar 1944 ausgestellt und tritt auf Guisans Befehl hin in Kraft. Darin wird die gleiche Aufstellung, wie die des Operationsbefehls 13 beschrieben, jedoch mit leichten Änderungen.
\begin{itemize}
    \item Zusätzlich zu der Aufstellung der Truppen aus dem Operationsbefehl 13 soll eine neue Sperrzone, 2 Flügelstellungen und 2 Armee-Reservegruppen geschaffen werden.
    \item Die Sperrzone schützt die Waadtländer- und Berneralpen im Westen und in den Urner- und Glarneralpen im Osten die bestehenden Abwehorganisationen.
    \item Die Flügelstellungen in St. Maurice und Sargans werden durch je 1 Division verstärkt und werden zum «Gr. Westflügel» und «Gr. Ostflügel».
    \item In der Zentralschweiz und im Mittelland werden 2 Armee-Reservegruppen gebildet.
\end{itemize}
\subsubsection{Operationsbefehl 18} \label{Operationsbefehl 18}
Für den Operationsbefehl 18 ist weder ein Ausstellungsdatum noch ein Auslöser bekannt. Darin werden die Tätigkeiten des Gros der Armee, des erste, zweiten und vierten Armeekorps, der Armeereserven, der 7. Division und kleineren Gruppen beschrieben. Die Verteidungung wurde wie folgt aufgeteilt:
\begin{itemize}
    \item Das Gros der Armee hält die Stellung entlang der Linie Rhein bei Saline Schweizerhalle – W. Gempen – W. Homburg – Erschwil – Hohe Winde – Mont Raimeux – N. Moutier – Moron – Vallon de la Trame – Cortebert – W. Pt. 1341 – Chasseral – Pr. 1247 – Portuis – Mont d’Amin – Vue des Alpes – Mont Racine – La Tourne – Noiraigue – Le Solitat – Pt. 1360 – Pt. 1433 – Les Cernets – Chasseron – Col des Etroits – Cole de l’Aiguillon – Mont Suchet – Bel Coster – Ballaigues – Dent de Vaulion – Mollendruz – Mont Tondre – Marchairuz – Promenthouse.
    \item Das erste Armeekorps stützt sich auf die Linie Aubonne - Promenthouse, während es in Richtung Moudon - Fribourg zwischen Lac Leman und Mont Aubert den Druchgang sperrt.
    \item Das zweite Armeekorps sperrt den Druchgang in Richtung Bern zwischen Mont Aubert und Moron.
    \item Das vierte Armeekorps sperrt den Druchgang in Richtung Solothurn und Oensingen - Olten zwischen Moron und Rhein.
    \item Die Armeereserven sind in 3 Gruppen unterteilt, welche jeweils in einem Kreis aufgestellt sind:
    \begin{itemize}
        \item Geb. Br. 12 hält die Stellung entlang der Linie Ulmiz – Cournillens – Portalban – La Sauge – Ulmiz.
        \item L. Br. 2 hält die Stellung entlang der Linie Düdigen – Alterswil – Rechthalten – La Roche – Corbieres – Vuippens – Posieux – Belfaux – Ottisberg – Düdigen.
        \item L. Br. 3 hält die Stellung entlang der Linie Wohlen – Bümpliz – Köniz – Schwarzenburg – Heitenried – Grossbösingen – Mühleberg – Aare bis Wohlen.
    \end{itemize}
    \item Die 7. Division hält die Stellung entland der Linie Windisch – Killwangen – Niederwil – Dintikon – Seon – Gränichen – Aarau – Aare bis Windisch.
    \item Die Gruppe Mot. 5. AK ad hoc mit Mot. Div ad hoc 14 und 15 hält die Stellung entlang der Linie Wohlen – Künten – Urdorf – Albisrieden – Hausen a.A. – Baar – Cham – Root – Beinwil – Geltwil – Brestenberg – Villmergen – Wohlen.
    \item Die Gruppe Mot. Rgt. 30 hält die Stellung entlang der Linie Hohenrain – Inwil – Rothenburg – Rain - Hohenrain.
    \item Die Gruppe Mot. Rgt. 32 hält die Stellung entlang der Linie Sursee – Willisau – Gettnau – Dagmersellen – Büron - Sursee.
\end{itemize}
\subsubsection{Operationsbefehl 19} \label{Operationsbefehl 19}
TBD

\subsubsection{Operationsbefehl 20} \label{Operationsbefehl 20}
TBD

\subsubsection{Operationsbefehl 21} \label{Operationsbefehl 21}
TBD

\subsubsection{Operationsbefehl 22} \label{Operationsbefehl 22}
TBD