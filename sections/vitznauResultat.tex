\documentclass[../main.tex]{subfiles}
Die im Jahre 1941 während 13 Monaten erbaute Festung Vitznau gehörte zu einer der vielen Festungen der Achsensperrung von Luzern-Stans westwärts bis zum Brünigpass. Ihre Stellung erlaubte zudem den Schutz des auf der gegenüberliegenden Seeseite befindenden Flugplatzes Buochs/Ennetbürgen zu gewährleisten. Zu Beginn ihrer Einsatzzeit waren die 150'000 Franken teuren Kanonen in der Lage eine Reichweite von 17,5 Kilometern abzudecken, mit einer neu entwickelten Spitzgranate konnte diese einige Jahre später sogar um 4 Kilometer erweitert werden. Die Kadenz betrug 15 Schuss pro Minute. Die Festung bot Platz für 337 Soldaten.
\paragraph{}1998 wurde die Festung durch die Einwohnergemeinde Vitznau und durch die Kooperation der Festung vom VBS abgekauft, nachdem dieses mit der Armeereform 95 deren Ausmusterung anordnete. Dabei wird sie nun bis heute nicht als reines Museum, sondern vielmehr als Erlebnis-Festung angepriesen und bietet für die Besucher neben Führungen sogar Übernachtungen an, welche ein oberflächliches Eintauchen in die Welt der Soldaten ermöglichen.
\paragraph{}Beim Besuch der Festung Vitznau entstanden viele Bilder, die in die Webseite eingeflossen sind. Das Produkt beinhaltet einen Bereich über die Festung Vitznau, in welchem mithilfe eines Planes durch die verschiedenen Räume navigiert werden kann. Zu den Räumen gibt es verschiedene Bilder und informative Texte.