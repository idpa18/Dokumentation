\documentclass[../main.tex]{subfiles}

Quellen
    Operationsbefehle
        admin.ch
    Internet
    Führung
Webseite
    Karte
VsCode, LaTeX, Laptops/MacBook

Bei der Recherche für unsere Arbeit stiessen wir immer wieder auf Zitate, welche von einem Dokument Names "Operationsbefehl Nr. 13" stammt.
Dies brachte uns zur Suche nach dem vollen Inhalt dieses Dokukenten und den 12 davor. Da wir auf unserer Suche im Internet nicht fündig wurden, beschlossen wir, eine Email an die Adresse, welche auf admin.ch angegeben ist, zu schreiben.
Nach kurzer Zeit erhielten wir einen Link zu einer temporären Ablage, welche die Operationsbefehle 1 - 20?? enthielt.

Ein Grossteil der Informationen, die in unserer Arbeit zu finden sind, stammen direkt aus den Operationsbefehlen, welche uns netterweise vom Eidgenössischen Departement für Verteidigung,
Bevölkerungsschutz und Sport VBS zur Verfügung gestellt wurden.
Andere Informationen stammen aus dem Internet oder von der Führung durch die Festung Vitznau, welche wir eines Samstagmorgens unter der Führung von Eich Steiner erhielten.

Die Operationsbefehle wurden uns in Form von eingesannten PDFs der originalen, mit der Schreibmaschine geschriebenen Dokumenten geschickt.
Das Lesen und anschliessende Zusammenfassen der Dokumente haben wir in der Gruppe aufgeteilt. Aus jedem Befehl wurden die Gruppen, deren Zusammensetzung und Position, Verschiebungen und generelle Informationen ausgelesen und in einem Word-Dokument zusammengefasst und in einer Karte eingezeichnet.

Die Dokumentation, die Sie gerade lesen, wurde mit LaTeX geschrieben.
3 unserer Gruppenmitglieder nutzten einen Windows-Laptop und ein Mitglied ein MacBook, bei der Wahl der Technologien musste also darauf geachtet werden, dass sie auf jedem Betriebssystem genutzt werden können.


