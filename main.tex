\documentclass[opensans, a4paper]{article}

\usepackage[utf8]{inputenc}
\usepackage[default]{opensans}
\usepackage[margin=2cm]{geometry}
\linespread{1.5}

\usepackage{setspace}
\usepackage[style=authoryear,citestyle=authoryear]{biblatex}
\usepackage{blindtext}
\usepackage{biblatex}
\addbibresource{bibliography/opbef.bib}

\usepackage{chngcntr}  % Counter für durchgängige Nummerierung
\counterwithout{figure}{section}
\counterwithout{table}{section}  

\usepackage[titles]{tocloft} %titles für seperate Seiten

\usepackage{graphicx}
\graphicspath{{images/}{../images/}}
 
\usepackage{subfiles}

\usepackage{hyperref}

\usepackage{mathpazo} % Palatino font

\usepackage[toc]{glossaries}
\subfile{sections/glossar}

% german specific
\usepackage[T1]{fontenc}
\usepackage[ngerman]{babel}
\usepackage{hyphenat} 

% styling for new paragraph
\setlength{\parindent}{0em}
\setlength{\parskip}{1em}
\renewcommand{\baselinestretch}{1.5}
\renewcommand{\thefigure}{\arabic{figure}}
\renewcommand{\cftfigpresnum}{Abb. }
\renewcommand{\cftfigaftersnum}{.}

\title{Eine wehrhafte Schweiz} 
\author{Jäger, Schmassmann, Wälti, Wigger}

\begin{document}

\subfile{sections/titelSeite}

\pagestyle{empty}
\tableofcontents
\clearpage
\pagestyle{plain}

\newpage

\section{Vorwort} \label{Vorwort}
\subfile{sections/vorwort}
\newpage 

\section{Abstract} \label{Abstract}
\subfile{sections/zusammenfassung}
\newpage

\section{Einleitung} \label{Einleitung}
\subfile{sections/einleitung}
\newpage

\section{Vorgehen} \label{Vorgehen}
\subfile{sections/vorgehen}
\newpage

\section{Resultate} \label{Resultate}
\subfile{sections/resultate}
\newpage

\section{Diskussion} \label{Diskussion}
\subfile{sections/diskussion}
\newpage

\addcontentsline{toc}{section}{Bibliographie}
\setlength{\bibhang}{0pt}
\nocite{*}
{ \setstretch{1.0}
\let\thefootnote\relax\footnotetext{Den von der Berufsbildung Baden geforderten Bibliographie-Leitfaden können wir leider nicht komplett einhalten, da unsere Arbeit mit LaTeX verfasst wurde, welches diesen Standard nicht unterstützt.}
\let\thefootnote\relax\footnotetext{Die Informationen der beiden Wikipedia-Quellen konnte nicht ausfindig gemacht werden, weshalb wir uns -- entgegen den Vorgaben der IDPA -- lediglich auf diese stützen.}
\let\itshape\upshape
\setlength\bibitemsep{10pt}
\printbibliography [
    title={Bibliographie}
]
\setlength\bibitemsep{0pt}
}
\newpage

\addcontentsline{toc}{section}{Abbildungsverzeichnis}
\renewcommand{\cftfigindent}{0pt}
\setlength{\cftfignumwidth}{1.4cm}
\let\thefootnote\relax\footnotetext{Den von der Berufsbildung Baden geforderten Bibliographie-Leitfaden können wir leider nicht komplett einhalten, da unsere Arbeit mit LaTeX verfasst wurde, welches diesen Standard nicht unterstützt.}
\listoffigures
\newpage

\section{Anhang} \label{Anhang}
\subfile{sections/anhang}

\end{document} 