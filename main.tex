\documentclass[opensans, a4paper]{article}

\usepackage[utf8]{inputenc}
\usepackage[default]{opensans}
\usepackage[margin=2cm]{geometry}
\linespread{1.5}

\usepackage{setspace}
\usepackage[style=authoryear,citestyle=authoryear]{biblatex}
\addbibresource{bibliography/opbef.bib}

\usepackage{chngcntr}  % Counter für durchgängige Nummerierung
\counterwithout{figure}{chapter}                                                
\counterwithout{table}{chapter}  

\usepackage[titles]{tocloft}                  %titles für seperate Seiten
\renewcommand{\thefigure}{\bfseries\arabic{figure}}
\renewcommand{\cftfigpresnum}{\textbf{Abb.}  }
\renewcommand{\cftfigaftersnum}{\textbf{.}}
\renewcommand{\cftdotsep}{\cftnodots}

\usepackage{graphicx}
\graphicspath{{images/}{../images/}}
 
\usepackage{subfiles}

\usepackage{hyperref} 

\usepackage{mathpazo} % Palatino font

\usepackage[toc]{glossaries}
\subfile{sections/glossar}

% german specific
\usepackage[T1]{fontenc}
\usepackage[ngerman]{babel}
\usepackage{hyphenat} 

% styling for new paragraph
\setlength{\parindent}{0em}
\setlength{\parskip}{1em}
\renewcommand{\baselinestretch}{1.5}

\title{Der Réduit-Plan} 
\author{Stefi, Joel, Seya und Dario}

\begin{document}

\subfile{sections/titelSeite}

\pagestyle{empty}
\tableofcontents
\clearpage
\pagestyle{plain}

\newpage

\section{Vorwort} \label{Vorwort}
\subfile{sections/vorwort}
\newpage 

\section{Zusammenfassung} \label{Zusammenfassung}
\subfile{sections/zusammenfassung}
\newpage

\section{Einleitung} \label{Einleitung}
\subfile{sections/einleitung}
\newpage

\section{Vorgehen} \label{Vorgehen}
\subfile{sections/vorgehen}
\newpage

\section{Resultate} \label{Resultate}
\subfile{sections/resultate}
\newpage

\section{Diskussion} \label{Diskussion}
\subfile{sections/diskussion}
\newpage

\section{Quellen} \label{Quellen}
{ \setstretch{1.0}
\footnote{Den von der Berufsbildung Baden geforderten Bibliographie-Leitfaden können wir leider nicht komplett einhalten, da unsere Arbeit mit LaTeX verfasst wurde, welches diesen Standard nicht unterstützt.}
\footnote{Die Informationen der beiden Wikipedia-Quellen konnte nicht ausfindig gemacht werden, weshalb wir uns - entgegen den Vorgaben der IDPA - lediglich auf diese stützen.}
\printbibliography 
}
\newpage

\section{Bilder} \label{Bilder}
\footnote{Den von der Berufsbildung Baden geforderten Bibliographie-Leitfaden können wir leider nicht komplett einhalten, da unsere Arbeit mit LaTeX verfasst wurde, welches diesen Standard nicht unterstützt.}
\listoffigures
\newpage

\section{Anhang} \label{Anhang}
\subfile{sections/anhang}

\end{document} 