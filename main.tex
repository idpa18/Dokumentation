\documentclass[opensans]{article}

\usepackage[utf8]{inputenc}
\usepackage[default]{opensans}

\usepackage{biblatex}
\addbibresource{bibliography/opbef}

\usepackage{graphicx}
\graphicspath{{images/}{../images/}}
 
\usepackage{subfiles}

\usepackage[toc]{glossaries}
\subfile{sections/glossar}

% german specific
\usepackage[T1]{fontenc}
\usepackage[ngerman]{babel}
\usepackage{hyphenat}
 
\title{Der Réduit-Plan}
%\subtitle{Die Schweizer Verteidigungsstrategie im zweiten Welt Krieg}
\author{Stefi, Joel, Seya und Dario}

\begin{document}

\begin{titlepage}
    \maketitle
\end{titlepage}

\tableofcontents

\newpage

\section{Test}
The \gls{latex} typesetting markup language

\section{Vorwort} \label{Vorwort}
\subfile{sections/vorwort}
\newpage 

\section{Zusammenfassung} \label{Zusammenfassung}
\subfile{sections/zusammenfassung}
\newpage

\section{Einleitung} \label{Einleitung}
\subfile{sections/einleitung}
\newpage

\section{Vorgehen} \label{Vorgehen}
\subfile{sections/vorgehen}
\newpage

\section{Resultate} \label{Resultate}
\subfile{sections/resultate}
\newpage

\section{Diskussion} \label{Diskussion}
\subfile{sections/diskussion}
\newpage

\section{Glossar} \label{Glossar}
\printglossaries
\newpage

\section{Quellen} \label{Quellen}
\printbibliography
\newpage

\section{Anhang} \label{Anhang}
\subfile{sections/anhang}
 
\end{document}